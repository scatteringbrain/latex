\documentclass{report}
\usepackage{graphicx}
\usepackage{amsmath}
\title{Arfken} 
\begin{document}
Exercises 4.1.1 \\ \\ 
Show that if all the components of any tensor of any rank vanish in one particular coordinate system, they vanish in all coordinate systems.\\
Note. This point takes on special importance in the four-dimensional (4-D) curved space of general relativity. if a quantity, expressed as a tensor, exists in one coordinate system, it exists in all coordinate systems and is not just a consequence of a choice of a coordinate system(as are centrifugal and Coriolis forces in Newtonian mechanics.)\\\\
Q. covariant four-vector?\\ \\
Scalars : Invariant under rotations of the coordinate system, Single number \\\\ 
Vectors : The number of real components = The dimension of the coordinate system. \\\\
A tensor of rank n in d dimension , components : $d^{n}$ \\ \\ 
\(\textbf{A} = A_{1}\mathbf{e_{1}} + A_{2}\mathbf{e_{2}} + A_{2}\mathbf{e_{2}} \) \\
\(\textbf{A}' = A'_{1}\mathbf{e'_{1}} + A'_{2}\mathbf{e'_{2}} + A'_{2}\mathbf{e'_{2}} \)\\
\(\mathbf{e_{i}} (i = 1, 2, 3)\) : Coordinate system, Before rotation \\
\(\mathbf{e'_{i}} (i = 1, 2, 3)\) : Coordinate system, After rotation \\
\textbf{A} = \textbf{A}' \\
Expressing the relation : \\
\( \textbf{A}'_{i} = \sum_{j}(\mathbf{e'_{i}} \cdot \mathbf{e_{j}}) \textbf{A}_{j} \) \\
\( \textbf{A}'_{i} = \sum_{j}( \frac {\partial x'_{i}} {\partial x_{j}}) \textbf{A}_{j} \) \\ 
\begin{figure}[h!]
  \includegraphics[width=50mm]{pic11.jpg}
  \label{fig:pic111}
\end{figure}\\ \\
This$(\textbf{A})$ is called a \textbf{contravariant} vector. \\ \\
The gradient of a scalar $\phi$ \\ \\
\( (\nabla \phi)'_{i} \equiv (\partial \phi / \partial x_{j}) \mathbf{e_{j}} \), \\ \\
\( (\nabla \phi)'_{i} \equiv \sum_{j} \frac{\partial x_{j}}{\partial x'_{i}} \frac{\partial \phi}{\partial x_{j}} \)\\

This$(\nabla \phi)$ is called a \textbf{covariant} vector. \\ \\



\end{document}