\documentclass{report}
\usepackage{graphicx}
\usepackage{amsmath}
\title{chapter5} 
\begin{document}
Problem 1. \\ \\ 
(a) The deutron's mass is 1875.6 MeV/$c^{2}$. What is its binding energy? is this a relativistic system? \\
(b) If you take the up- and down-quark masses to be those given in Table 4.4, what is the binding energy of a pion? is this a relativistic system? \\\\
Solution\\
(a) One neutron + One proton - 1875.6 MeV/$c^{2}$ \\ 
Neutron : 939.565 MeV/$c^{2}$\\ 
Proton : 938.272 MeV/$c^{2}$\\
Mass difference = 2.2 MeV/$c^{2}$\\\\
Q. A relativistic system? \\
A. mass difference $\times$ $c^{2}$ = binding energy = 2.2 MeV (in wikipedia, it says 2.2 MeV, yes)\\ \\
(b)  
 \begin{figure}[h!]
  \includegraphics[width=\linewidth]{Table4.4.png}
  \caption{Table4.4.}
  \label{fig:Table4.4}
\end{figure}
Figure \ref{fig:Table4.4} Quark masses.

\[M(meson) = m_{1} + m_{2} + A \frac{(S_{1} \cdot S_{2})}{m_{1}m_{2}} \ \ (5.48) \]\\
This is because of the fact that the pseudoscalr and vector mesons have different masses even though they have the same quark contents while  their spin orientations are different. \\ 
\begin{equation}
\textbf{$S_{1}$} \cdot \textbf{$S_{2}$} = \frac{1}{2}(S^{2} -S_1^{2} - S_2^{2}) =\left\{
\begin{array}{@{}rl@{}}
 \frac{1}{4} \hbar^2, \ \textnormal{for} \ s = 1 \ (\textnormal{vector mesons}) \\
- \frac{3}{4} \hbar^2, \ \textnormal{for} \ s =   0 \ (\textnormal{pseudoscalars})
\end{array}
\right .
\end{equation}    \\
% ref : https://tex.stackexchange.com/questions/283013/how-to-get-big-curly-braces-for-an-equation-with-3-parts
$m_{u}$ = $m_{d}$ = 308 MeV/$c^{2}$, $m_{s}$ = 483 MeV/$c^{2}$,  A = 159 MeV/$c^{2}$ \\

\begin{equation}
\left\{
\begin{array}{@{}ll@{}}
|1 \ 1\rangle = -u \bar{d} \\
|1 \ 0 \rangle = (u \bar{u} - d \bar{d} )/ \sqrt{2} \\
|1 \ -1\rangle = d \bar{u} \\
\end{array}
\right .
\end{equation} 
% ref : https://tex.stackexchange.com/questions/61481/left-aligning-the-conditional-statement-inside-an-array
\(\pi^+ = u + \bar{d} = 2 MeV/c^2+ 5 MeV/c^2 + 159 MeV/c^2 \frac{(197\times197)MeV^2 fm^2}{10 MeV^2/c^4} = 7 MeV/c^2 + 159\times197\times197\times10^{-30}\times10^{16}\times \frac{1}{10} MeV/c^2 = 7 MeV/c^2 + 6.17\times 10^{-9} MeV/c^2\) \\
The mass of \(\pi^+\) is 139 \(MeV/c^2\) .(wiki-pion) \\
Therefore the binding energy is approximately 139 - 7 = 132 \(MeV/c^2\). \\

\end{document}