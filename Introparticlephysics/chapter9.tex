\documentclass{report}
\usepackage{graphicx}
%\usepackage{tipk}
%\usepackage{tikz-feynman}
\usepackage[compat=1.0.0]{tikz-feynman}
\usepackage{amsmath}
\title{chapter9} 
\begin{document}
Problem 1. \\ \\ 
Derive the completeness relation for a massive particle of spin 1 (see Problem 9.27 for the massless analog.) [Hint: Let the z axis point along \textbf{p}. First construct three mutually orthogonal polarization vectors ($\epsilon^{(1)}_{u}$,$\epsilon^{(2)}_{u}$,$\epsilon^{(3)}_{u}$) that satisfy $p^u$$\epsilon_u$ = 0 $\epsilon_u$$\epsilon^u$ = -1.] \\ \\ \\
Solution \\

First, considering the Lorentz condition : \(\partial_u A^u = 0 \) (7.82) \\
For photon's case, which has spin 1 but massless, 
\[ A^\mu = a e^{-(i/\hbar)p \cdot x} \epsilon^\mu (p)  \ \ \ \ \ (7.89)\]\\
This should be the same. but p(momentum) will be different, because of the mass. \\ 
1) The polarization vectors $\epsilon_\mu^{(s)}$ should satisfy the momentum space Lorentz and othorgonal conditions  (7.101) $\sim$ (7.103): \\
\(p^{\mu}\epsilon_{\mu} = 0, \ \ \ \epsilon^{(1)}_{\mu} \epsilon^{(2)\mu} = 0, \ \ \ \epsilon^{(1)}_{\mu} \epsilon^{(1)\mu} = -1 \)\\ 
2) Using the Lorentz condition : \(\epsilon^{\mu}p_{\mu} = 0  \ \ (9.2) \), \ and set the z axis as a polarization, $\epsilon^{(1)}_{\mu}$ = (1, 0, 0) and $\epsilon^{(2)}_{\mu}$ = (0, 1, 0),\\

%A massive particle of spin 1 has three allowed states ( $m_s$ = 1, 0, -1)




%Q. what is the completeness relation? \\
%A. if the limit n $\rightarrow$ $\infty$ and the coefficient $c_n$ of the wavefunction $\psi$ converges to 0,  the formula, 
%\[ \int \vert \psi \vert ^2 d^3r = \sum_{i=0}^{\infty} \vert c_i \vert^2 , \textnormal { merzbacher (4.29)} \] \\
%This equation is the completeness relation. This means that set of the function is enough in number of states to express any arbitrary states. \\ 

%Q. How to know the completeness relation for a massive spin 1 particle?\\
%A. Since the complete relation says that the integral of the squares of the state is the sum of the squares of the coefficients, and using the Hint, integrate the vectors' dot products.\\

%Q. where does the polarization vectors come from? \\
%A. Spin 1 particles are bosons, and $\epsilon^u$ is the polarization vectors to the direction of the propagation. The number of vectors is differ to photon, gluon, W and Z bosons. \\ 

%Q. in QED, the photon's polarization vector was evaluated by the maxwell equation and lorentz invariance, but how does that for gluon, W and Z bosons? \\ 
%A. For quark 'states' in QCD, it needs additional 'color states'. And Dirac spinors of the QED is included to QCD quark states. \\
%So, in QCD, the External lines are \( u^{(s)}(p)c \). where s : spin, p : momentum, c : color. \\ 
%I don't know about W and Z. \\\\ 

%However, back to the problem, the massive spin 1 particle, the charge is not given, \\ 

%Q. Why a massive particle in spin 1 has 3 states? when a massless particle has only 2? \\ 
%A. in the page 309 of the textbook annotation, the number of degrees of freedom for M $\neq$ 0 is 2s + 1, while 2 for M = 0 . \\ 




\end{document}