\documentclass{report}
\usepackage{graphicx}
\usepackage[compat=1.0.0]{tikz-feynman}
\usepackage{amsmath}
\title{chapter10} 
\begin{document}
Problem 1. \\ \\ 
One advantage of the Lagrangian formulation is that it does not commit us to any particular coordinate system - the q's in Equation 10.6 could be Cartesian coordinates, or polar coordinates, or any other variables we might use to designate the particle's position. Suppose, for example, we want to analyze the motion of a particle that slides frictionlessly on the inside surface of a cone mounted with its axis pointing upward, as shown.\\ 
\end{document}