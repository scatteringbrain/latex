\documentclass{report}
\usepackage{graphicx}
%\usepackage{tikz-feynman}
\usepackage[compat=1.0.0]{tikz-feynman}
\usepackage{amsmath}
\title{chapter8} 
\begin{document}
Problem 1. \\ \\ 
(a) Derive Equation 8.1, from the Feynman rules for QED.\\
(b) Obtain Equation 8.2 from Equation 8.1\\
(c) Derive Equation 8.3 from Equation 8.2\\
(d) Derive Equation 8.4 from Equation 8.3\\ 

{\center  $\textnormal{Feynman Rules}$  \\ } 
1. Notation : \\
external lines with momentum $p_1, p_2 ...$ , arrow direction forwards in time \\
internal lines with momentum $q_1, q_2 ...$  , arrow direction forwards in time \\ \\

2. External lines : 
\begin{center}
\textnormal{Electrons} 
	\begin{cases}
		%\textnormal{if bank issues ABs at time }\\ 
	\end{cases}
\end{center}
3. Vertex factors (where lines meet) :  \\
\[ ig_e\gamma^\mu \] \\ \\

4. propagator : each internal lines' factors \\ 
{\center $\textnormal{Electrons and positrons}$ : \(\frac{i(\gamma^{\mu}q_{\mu} + mc)}{q^{2}-m^{2}c^{2}}\) \\  }
{\center $\textnormal{Photons}$ : \(\frac{-ig_{\mu\nu}}{q^{2}}\) \\  }
%$\frac{i($\gamma^{\mu}$$q_{\mu}$ + mc)}{$q^{2}$-$m^{2}$$c^{2}$}$
%\(\frac{$\gamma^u$$q_u$ + mc}{$q^2$-$m^2c^2$}\)
%$\frac{i($\gamma$$^u$$q_u$ + mc}{$q^2$-$m^2$$c^2$}$
%{\center $\textnormal{Electrons and positrons}$ : $\frac{i($\gamma^u$$q_u$ + mc)}{$q^2$-$m^2$$c^2$}$\\}
\vspace{10mm}

5. Conservation of E and P for each vertex : \\ 
{\center $(2\pi)^{4}$ $\delta^{4}$($k_1$+$k_2$+$k_3$) \\ }
\vspace{5mm}
where k's are the three four-momenta \\ 

6. Integrate over internal momenta : For internal moment q, the factors are  
{\center \(\frac{d^{4}q}{(2\pi)^4}\)  \\ }
And then integrate \\ \\ 

7. Cancel the delta function : 
{\center \((2\pi)^4\delta^4(p_1+p_2 +... - p_n)\)  \\ }
\vspace{5mm}
This will simplify the integral $\rightarrow$ only one of q left by the conservation and return 1 by the delta function. \\  \\ \\ \\
Solution \\ \\
\begin{equation}
\feynmandiagram [horizontal=a to b] {
i1 [particle=\(e^{-}\)] -- [fermion, momentum'=\(p_1\)] a -- [fermion, rmomentum'=\(p_2\)] i2 [particle=\(e^{+}\)],
a -- [photon, momentum'=\(q\)] b,
f1 [particle=\(\mu^{-}\)] -- [anti fermion, rmomentum'=\(p_3\)] b -- [anti fermion, momentum'=\(p_4\)] f2 [particle=\(\mu^{+}\)],
};
\end{equation}
A transition amplitude for the first order of perturbation quantum-mechanics, peskin eqn(1.2), \\ 
In QM perturbation theory, to first order, the amplitude is, \\
\begin{center}
$\langle$ $\textnormal{final state}$ \(|\ H_1\ |\) $\textnormal{initial state}$ $\rangle$ 
\end{center} 
This is the first order, but the hamiltonian can not mediate the two state, but gamma does it.\\
So, expand this equation to the next order with $\gamma$.\\ \\

For ($e^-$ + $e^+$ $\rightarrow$ $\mu^-$ + $\mu^+$), \\
\begin{center}
M $\sim$ $\langle$$u^+$$u^-$ $| H_1 |$ $\gamma$ $\rangle$$^u$ $\langle$$\gamma$$| H_1 |$$e^+$$e^-$$\rangle$$_u$ \\ 
\end{center}
1. External electron lines : $|$$e^{+}e^{-}$$\rangle$\\
2. External muon lines : $\langle$$e^{+}e^{-}$$|$\\
3. The vertices : $H_1$ = \( ig_e\gamma^\mu \) always for QED. The object $\gamma^\mu$ are 4 x 4 matrices. \\
4. Internal photon line : $|$$\gamma$$\rangle$$\langle$$\gamma$$|$ =  \(\frac{-ig_{\mu\nu}}{q^{2}}\)  \\
\begin{center}
 $\langle$$\textnormal{final state}$ | $H_1$ | $\textnormal{initial state}$$\rangle$\\ 
\end{center}
(small notes for myself : \\ 
Q. why do we only use spinors ?\\ 
A. chapter7.2 The solution for dirac equation \\
Q. General solution of QM \\
A. Get the E.S and E.V from assuming p = 0 $\rightarrow$
Get the solutions for the general p with the states when p = 0) \\ 

\end{document}