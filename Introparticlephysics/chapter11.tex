\documentclass{report}
\usepackage{graphicx}
\usepackage[compat=1.0.0]{tikz-feynman}
\usepackage{amsmath}
\usepackage{sansmath}
\usepackage{tikz}
\usepackage{tkz-euclide}
\usepackage{tikz-3dplot}
\usepackage[makeroom]{cancel}
\usetikzlibrary{shadings,intersections,patterns}
\usepackage{setspace}
\doublespacing
\title{chapter11} 
\begin{document}

Problem 1. \\ \\ 
Estimate the lifetime of the sun, assuming (as Lord Kelvin did) that the source of the energy radiated is gravity. Look up any empirical numbers (the power radiated by the sun, the mass, and radius of the sun).\\\\
Solution\\
From wiki - kelvin-Helmholtz mechanism,\\
\(U = -\frac{Gm_1m_2}{r}\),\\
\(U = -G \int_{0}^{R}\frac{m(r)4\pi r^2 \rho}{r}dr = -G \int_{0}^{R}\frac{4\pi r^3 4\pi r^2 \rho^2}{r}dr = - \frac{16}{15}G\pi^2\rho^2R^5 = - \frac{3GM^2}{5R} \)\\
Using the Virial theorem, \(U_r = \frac{|\langle U \rangle |}{2} = \frac{3GM^2}{10R} \) \\
And divide the potential energy by the luminosity of the Sun, \(\frac{U_r}{L_\odot} \approx \frac{1.1 \times 10^{41} J} {3.828 \times 10^26 W} = 2.874 \times 10^{14} s \approx 89000000 \ \ \textnormal{years}\).\\ \\ \\ \\

Problem 2. \\ \\
(a) What is the period of $K^0 = \bar{K}^0 oscillations (Section 4.4.3)$? [Hint:The mass eigenstates are $K_S^0$ and $K_L^0$. In the neutrino case (Eq. 11.7) the particles were highly relativistic; for the K's, assume on the contrary that the kinetic energy is substantially less than the rest energy.]\\
(b) Compare your result in (a) to the lifetime of $K_S^0$ and $K_L^0$. Notice that the $K_S^0$ component of the beam dies out - leaving pure $K_L^0$ - well before significant oscillation can occur. \\\\

Solution \\ 
(a) \\
Q. What is the `mass eigentstates'?\\ 
A. From textbook p.390 "If one can spontaneously convert into the other, it means that neither is an eigenfunction of the Hamiltonian. The true stationary states for the system are evidently some orthogonal linear combinations." \\\\
Q. What are the $K_L^0$ and $K_S^0$? \\
A. $K^0$ with strangeness +1, can turn into its antiparticle $\bar{K}^0$ with strangeness -1. The normalized eigenstates of CP for the neutral kaon are, \\ \\
\[ |K_1\rangle = (\frac{1}{\sqrt{2}})(|K^0\rangle - |\bar{K}^0\rangle) \ \textnormal{and} \ |K_2\rangle = (\frac{1}{\sqrt{2}})(|K^0\rangle + |\bar{K}^0\rangle) \ \ \ \ \ (4.68) \]\\
Here, let $K_1$ and $K_2$ to be $K_L^0$ and $K_S^0$.\\
\[ |K_L^0\rangle = (\frac{1}{\sqrt{2}})(|K^0\rangle - |\bar{K}^0\rangle) \ \textnormal{and} \ |K_S^0\rangle = (\frac{1}{\sqrt{2}})(|K^0\rangle + |\bar{K}^0\rangle) \ \ \ \ \ \]\\
And \( |K^0\rangle = (\frac{1}{\sqrt{2}})(|K_1\rangle + |K_2\rangle) =  (\frac{1}{\sqrt{2}})(|K_L^0\rangle + |K_S^0\rangle)\)\\ 
And \( |\bar{K}^0\rangle = (\frac{1}{\sqrt{2}})(|K_1\rangle - |K_2\rangle) =  (\frac{1}{\sqrt{2}})(|K_S^0\rangle - |K_L^0\rangle)\)\\ 
Now, to get the period of the oscillation, 
\end{document}