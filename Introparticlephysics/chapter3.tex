\documentclass{report}
\usepackage{graphicx}
\title{chapter3} 
\begin{document}
Problem 1. \\ \\ 
Solve Equation 3.1 for x, y, z, t in terms of x', y', z', t', and check that you recover Equation 3.3. \\ \\

Solution \\ \\
Equation 3.1 \\\\
i. x' = $\gamma$(x - vt)\\
ii. y' = y\\
iii. z' = z\\
iv. t' = $\gamma$(t - $\frac{v}{c^{2}}$x)\\ \\
Equation 3.3 \\\\
i'. x = $\gamma$(x' + vt')\\
ii'. y = y' \\
iii'. z = z' \\
iv'. t = $\gamma$(t' + $\frac{v}{c^{2}}$x')\\ \\

i', ii' , iii' are obvious. \\ 

for iv', from iv, \\\\ t = $\frac{v}{c^{2}}$x + $\frac{t'}{\gamma}$ \\
= $\frac{1}{\gamma}$(t' + $\frac{\gamma}{c^{2}}$v($\frac{x'}{\gamma}$+vt))\\
 = $\frac{1}{\gamma}$(t' + $\frac{\gamma}{c^{2}}$v($\frac{x'}{\gamma}$)+$\frac{v^{2}\gamma t}{c^{2}}$)\\ \\
t(1-$\frac{v^{2}}{c^{2}}$) = $\frac{1}{\gamma}$(t'+$\frac{vx'}{c^{2}}$) \\ \\
Therefore,   t = $\gamma$(t'+$\frac{vx'}{c^{2}}$)\\\\
This is the iv'. \\ \\ \\

Problem 2. \\ \\ 
(a) Derive Equation 3.4. \\
(b) According to clocks on the ground (system S), streetlights A and B (situated 4km apart) were both turned on at precisely 8:00 P.M. which one went on first according to an observer on a train (system S'), which moves from A toward B at $\frac{3}{5}$ the speed of light? How much later (in seconds) did the other light go on? Note: As always in relativity, we are talking here about what S' observed, after correcting for the time it took the light to reach her, not what she actually saw ( which would depend on where she was located on the train). \\ \\

Solution for (a)\\ \\

From Equation 3.1 iv. t' = $\gamma$(t-$\frac{v}{c^{2}}$x), \\ \\
$t'_{a}$ =  $\gamma$($t_{a}$-$\frac{v}{c^{2}}$ $x_{a}$),\\ \\
$t'_{b}$ =  $\gamma$($t_{b}$-$\frac{v}{c^{2}}$ $x_{b}$),\\ \\
$t'_{b}$ - $t'_{a}$ = $\gamma$($t_{b}$ - $t_{a}$ - $\frac{v}{c^{2}}$ $x_{a}$ + $\frac{v}{c^{2}}$ $x_{b}$)\\ \\
Here, $t_{b}$ = $t_{a}$ by the condition, and $x_{a}$ $\neq$ $x_{b}$, \\ \\ 
So, $t'_{a}$ = $t'_{b}$ + $\frac{\gamma v}{c^{2}}$($x_{b}$ - $x_{a}$) \\ \\ 

Solution for (b) \\ \\ 
1) The streetlight B will go on first. \\
2) $t'_{a}$ = $t'_{b}$ + $\frac{\gamma v}{c^{2}}$($x_{b}$ - $x_{a}$) from (a),\\
\[\gamma = \frac{1}{\sqrt{1-\frac{v^{2}}{c^{2}}}}\]
v = $\frac{3}{5}$c $\rightarrow$ $\gamma$ = $\frac{\sqrt{5}}{2}$ \\
\[t'_{b} - t'_{a} = \frac{\sqrt{5}}{2}\frac{3}{5} \frac{4\times10^{3}}{3\times10^{8}}s\] \\ \\ \\ \\
Problem 3.3\\ \\
(a) How do volumes transform? (If a container has volume V' in its own rest frame, S', what is its volume as measured by an observer in S, with respect to which it is moving at speed v?) \\
(b) How do densities transform? (If a container holds $\rho'$ molecules per unit volume does it carry in S?)\\\\
Solution\\


Problem 3.8\\ \\ 
Show that the quantity I (in Equation 3.13) is invariant under Lorentz transformations (Equation 3.8). \\ \\

Solution \\ \\ 
$(x^{0'})^{2}$ = $\gamma^{2}$$((x^{0})^{2}+\beta^{2}(x^{1})^{2}-2\beta x^0 x^1)$\\
$(x^{1'})^{2}$ = $\gamma^{2}$$((x^{1})^{2}+\beta^{2}(x^{0})^{2}-2\beta x^0 x^1)$\\
$(x^{2'})^{2}$ = ${x^{2}}^{2}$\\
$(x^{3'})^{2}$ = ${x^{3}}^{2}$\\
$(x^{0'})^{2}$ -$(x^{1'})^{2}$-$(x^{2'})^{2}$-$(x^{3'})^{2}$ = $\gamma^{2}$(1 - $\beta^{2}$)($(x^{0})^{2}$-$(x^{1})^{2}$) - $(x^{2})^{2}$ -  $(x^{3})^{2}$ \\
=  $(x^{0})^{2}$ - $(x^{1})^{2}$ - $(x^{2})^{2}$ - $(x^{3})^{2}$\\


\end{document}