\documentclass{report}
\usepackage{graphicx}
\usepackage{amsmath}
\title{chapter7} 
\begin{document}

Problem 1. \\ \\ 
Show that \( \partial \phi / \partial x^{\mu} \) is a covariant four-vector ($\phi$ is a scalar function of x, y, z, and t).\\
Hint: First determine (from Equation 3.8) how covariant four-vectors transform; then use  
$\partial \phi$ / $\partial x^{\mu'}$ =  ($\partial \phi$ / $\partial x^{\nu}$)($\partial x^{\nu}$ / $\partial x^{\mu'}$) transforms. \\ \\
Solution \\ \\
The four-vectors in x, y, z, t : \\  $x^{0}$ = $ct$,\ $x^{1}$ = $x$,\ $x^{2}$ = $y$,\ $x^{3}$ = $z$. Equation (3.7)\\ \\
Equation (3.8) \\ \\
$x^{0'}$ = $\gamma$($x^{0}$ - $\beta$$x^{1}$) \\
$x^{1'}$ = $\gamma$($x^{1}$ - $\beta$$x^{0}$) \\
$x^{2'}$ = $x^{2}$ \\
$x^{3'}$ = $x^{3}$ \\
\(x^{\mu '} = \sum_{\nu=0}^{3} \Lambda_\nu^\mu x^\nu \  (\nu = 0, 1, 2, 3)  = \Lambda_\nu^\mu x^\nu \) \\ \\
This is a contravariant tensor.\\ \\
A contravariant tensor: axes changes $\rightarrow$ the components $\textbf{inversly}$ change. And the vector does not change.\\ \\ 
This is why it is called contravariant tensor. \\ \\
A covariant : axes changes $\rightarrow$  the components also changes in the same way : vector $\rightarrow$  scalar. \\ \\ 
A covariant vectors' components change the same as coordinates' change.\\ \\
From Equation 3.8,  \\ \\
$x^{0'}$ = $\gamma$($x^{0}$ - $\beta$$x^{1}$)  $\rightarrow$  $x'_{0}$ = $\gamma$($x^{0}$ + $\beta$$x^{1}$) \\ \\ 
(The details are in Equation 3.1 and 3.3 , 3.3 is the inverse transform of 3.1. using the same inverse transform here.) \\\\ 
$x^{1'}$ = $\gamma$($x^{1}$ - $\beta$$x^{0}$)  $\rightarrow$  $x'_{1}$ = $\gamma$($x^{1}$ + $\beta$$x^{0}$) \\ \\ 
$x'_2 = x^{2} $ \\ \\
$x'_3 = x^{3} $ \\ \\
These are the covariant transform. \\ \\
Q. What is the scalar function $\phi$? \\ \\
The quantity I (in Equation 3.13) and \(r^{2} = x^{2} + y^{2} + z^{2}\) can be scalar function. (not sure) \\ \\
So, using, I = $(x^{0'})^{2}$ -$(x^{1'})^{2}$-$(x^{2'})^{2}$-$(x^{3'})^{2}$, \\ \\  
Since $I$ = $x_{\mu}$$x^{\mu}$, \\ \\
$\partial \phi$ / $\partial x^{\mu'}$ = $x_{\mu}$, covariant tensor.\\\

\textnormal{https://en.wikipedia.org/wiki/covariance\_and\_contravariance\_of\_vectors}\\

Problem 2 \\ 
7.2 Show that Equation 7.17 satisfies Equation 7.15 \\ 

Solution \\ 

Equation 7.15 : \(\{ \gamma^{\mu} , \ \gamma^{\nu} \} = 2g^{\mu \nu} \) \\ 

Equation 7.17 :  \(\gamma^0 = \begin{pmatrix} 1 & 0\\ 0 & -1 \end{pmatrix}, \gamma^i =  \begin{pmatrix} 0 & \sigma^i \\ -\sigma^i & 0 \end{pmatrix} \) \\
But, here, $1$ denotes the 2 $\times$ 2 unit matrix, and $0$ is the 2 $\times$ 2 matrix of zeros.\\ \\
So,   \(\gamma^0 = \begin{pmatrix} 1 & 0 & 0 & 0\\ 0 & 1 & 0 & 0 \\ 0 & 0 & -1 & 0 \\ 0 & 0 & 0 & -1 \end{pmatrix}, \gamma^i =  \begin{pmatrix} 0 & 0 & \sigma^i & 0 \\ 0 & 0 & 0 & \sigma^i \\  -\sigma^i & 0 & 0 & 0 \\ 0 & -\sigma^i & 0 & 0  \end{pmatrix} \) \\ \\
Anticommutator $\{ A,  B\}$ = AB + BA \\  

So,\vspace{5 mm} \\
\(\{ \gamma^0 , \gamma^i \} = \gamma^0 \gamma^i + \gamma^i \gamma0 = \) \vspace{5 mm}\\
\( \begin{pmatrix} 1 & 0 & 0 & 0\\ 0 & 1 & 0 & 0 \\ 0 & 0 & -1 & 0 \\ 0 & 0 & 0 & -1 \end{pmatrix} \begin{pmatrix} 0 & 0 & \sigma^i & 0 \\ 0 & 0 & 0 & \sigma^i \\  -\sigma^i & 0 & 0 & 0 \\ 0 & -\sigma^i & 0 & 0  \end{pmatrix} +  \begin{pmatrix} 0 & 0 & \sigma^i & 0 \\ 0 & 0 & 0 & \sigma^i \\  -\sigma^i & 0 & 0 & 0 \\ 0 & -\sigma^i & 0 & 0  \end{pmatrix} \begin{pmatrix} 1 & 0 & 0 & 0\\ 0 & 1 & 0 & 0 \\ 0 & 0 & -1 & 0 \\ 0 & 0 & 0 & -1 \end{pmatrix}  \) \vspace{5 mm} \\
= \(\begin{pmatrix} 0 & 0 &  \sigma^i & 0 \\ 0 & 0 & 0 & \sigma^i \\ \sigma^i & 0 & 0 & 0 \\ 0 & \sigma^i & 0 & 0 \end{pmatrix} + \begin{pmatrix} 0 & 0 & -\sigma^i & 0 \\ 0 & 0 & 0 & -\sigma^i \\ -\sigma^i & 0 & 0 & 0 \\ 0 & -\sigma & 0 & 0  \end{pmatrix} = \begin{pmatrix} 0 & 0 & 0 &0  \\ 0 & 0 & 0 & 0 \\ 0 & 0 & 0 & 0 \\ 0 & 0 & 0 & 0  \end{pmatrix} \) \vspace{5 mm}  \\
So, is this $2g^{0i}$ ? \\
Equation (3.14) \( g = \begin{pmatrix} 1 & 0 & 0 & 0 \\ 0 & -1 & 0 & 0 \\ 0 & 0 & -1 & 0 \\ 0 & 0 & 0 & -1  \end{pmatrix}\) \\\\
This is the minkowski metric, $g_{\mu\nu}$.\\ 
$g^{\mu\nu}$ is simply $g^{-1}$, and $g$ = $g^{-1}$ (in a flat space time) (from wiki-minkowski metric)  \\ 
However except $g^{ii}$ or $g^{00}$ every g is zero. \\ 






\end{document}