\documentclass{report}
\usepackage{graphicx}
\usepackage{tikz-feynman} 
\title{chapter2} 
\begin{document}

Problem 2.1 \\ \\
Calculate the ratio of the gravitational attraction to the electrical repulsion between two stationary electrons. (Do I need to tell you how far apart they are?) \\

solution \\ 

\[F_{g} = \frac{G m_{e}^{2}}{r^{2}}\] \\  
\[F_{e} = \frac{k q_{e}^{2}}{r^{2}}\] \\  
\[F_{g}/F_{e} = \frac{G m_{e}^{2}}{k q_{e}^{2}}\] \\  
\[G = 6.67 \times 10^{-11} m^{3}\cdot kg^{-1}\cdot s^{-2} , m_{e} = 9.1 \times 10^{-31} kg,\]
\[k = 8.98 \times 10^{9} N\cdot m^{2} \cdot C^{-2}, q_{e} = 1.6 \times 10^{-19} C\]  \\ 
\[F_{g}/F_{e} = 2.4 \times 10^{-43}\] \\

No, the distance squares are canceled out. \\ \\ 

Problem 2.2 \\ \\ 
Sketch the lowest-order Feynman diagram represneting Delbruck scattering: \(\gamma\) + \(\gamma\) \(\rightarrow\) \(\gamma\) + \(\gamma\). (This process, the scattering of light by light, has no analog in classical electrodynamics.)\\

Solution \\ \\ 
\feynmandiagram[scale=2,transform shape] [layered layout, horizontal=a to b] {
% Draw the top and bottom lines
i1 [particle=$\gamma$]
-- [photon] a
-- [fermion, edge label=\(e^{-}\)] b
-- [photon] f1 [particle=$\gamma$],
i2 [particle=$\gamma$]
-- [photon] c
-- [anti fermion, edge label=\(e^{-}\)] d
-- [photon] f2 [particle=$\gamma$],
% Draw the two internal fermion lines
{ [same layer] c -- [fermion, edge label=\(e^{-}\)] a },
{ [same layer] d -- [anti fermion, edge label=\(e^{-}\)] b},
};
%\begin{figure}
 % \includegraphics[width=\linewidth]{pp2.2.png}
 % \caption{The lowest photon scattering (not sure)}
%  \label{fig:Delbruck scattering}
%\end{figure} 

. \\ \\ \\ \\ \\

Problem 2.3 \\ \\
Draw all the fourth-order (four vertex) diagrams for Compton scattering. (There are 17 of them; disconnected diagrams don't count.)\\

Solution \\ 
\feynmandiagram[scale=2,transform shape] [layered layout, horizontal=a to b] {
% Draw the top and bottom lines
i1 [particle=\(e^{-}\)]
-- [fermion] a
-- [photon] b
-- [fermion] f1 [particle=\(e^{-}\)],
i2 [particle=$\gamma$]
-- [photon] c
-- [fermion, edge label=\(e^{-}\)] d
-- [photon] f2 [particle=$\gamma$],
% Draw the two internal fermion lines
{ [same layer] c -- [fermion, edge label=\(e^{-}\)] a },
{ [same layer] d -- [fermion, edge label=\(e^{-}\)] b},
};
(to be continued)\\\\

Problem 2.4 \\
Determine the mass of the virtual photon in each of the lowest-order diagrams for Bhabha scattering (assume the electron and positron are at rest). What is its velocity?\\
(Note that these answers would be impossible for real photons.)\\\\

Solution\\
The mass of the virtual photon should be the mass of the electron + the positron.\\
2 $\times$ 0.510 $MeV$. \\
The velocity would be the vector sum of the electron and positron's velocity by the momentum conservation.\\ 
Since the two particles were at rest, the virtual photon's velocity will be zero.\\ 








\end{document}