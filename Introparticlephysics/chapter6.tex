\documentclass{report}
\usepackage{graphicx}
\usepackage{amsmath}
\title{chapter6} 
\begin{document}
Problem 1. \\ \\ 
Derive Equation 6.3. [Hint: What fraction of the original sample decays between t and t + dt? What, is the (initial) probability, p(t)dt, of any given particle decaying between t and t + dt? The average lifetime is \(\int_0^\infty t p(t) \,dt \).] \\ \\
Solution\\ \\
$\tau$ = $\frac{1}{\Gamma}$ ,  Equation (6.3) \\ \\
%Q. What is the probability? \\
%A. $\frac{N(t)}{N(0)}$? \\
Q. What fraction of the original sample decays between t and t + dt? \\ \\
A.  dN = -$\Gamma$Ndt, Equation (6.1) \\ \\
Q. What is the initial probability, p(t)dt? \\ \\
A. N(t) = N(0)$e^{-\Gamma t}$, Equation (6.2) \\ \\
Using probability density function, (from wiki), \\ \\
1 = C \(\int_0^\infty p(t) \,dt \) = C \(\int_0^\infty N(0)e^{-\Gamma t} \,dt \) , where C is normalizing coefficient.\\ \\
= C \(\cdot \frac{N_{0}}{\Gamma} \) \\ \\ 
C = \(\frac{\Gamma}{N_{0}}\) \\ \\
Q. The average lifetime? \\ \\
A. \(\tau = \langle t \rangle  = \int_0^\infty t p(t) \,dt  = \int_0^\infty t\frac{\Gamma}{N_{0}} N(0)e^{-\Gamma t} \,dt = \Gamma \int_0^\infty t e^{-\Gamma t} \,dt \)   \\\\
\(\Gamma \int_0^\infty t e^{-\Gamma t} \,dt = (integration \ by \ parts) = \frac{1}{\Gamma}  =\tau \) \\ \\ 
reference [wiki - Exponential decay, Probability Density function, integration by parts] \\ \\ \\ \\

Problem 2.\\
Nuclear physicist traditionally work with `half-life' ($t_{1/2}$) instead of mean life ($\tau$); $t_{1/2}$ is the time it takes for half the members of a large sample to decay. For exponential decay(Equation 6.2), derive the formula for $t_{1/2}$ (as a multiple of $\tau$). \\ \\

Solution \\ 
\( dN = - \Gamma Ndt \)\\
\( N(t) = N(0)e^{-\Gamma t} \) \\ 
Where $\Gamma$ is the decay rate, -$\Gamma$Ndt is the Number of decayed particles. \\ 
The mean lifetime is the reciprocal of the decay rate. \\
\(\tau = \frac{1}{\Gamma}\)\\ 



\end{document}