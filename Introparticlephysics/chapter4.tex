\documentclass{report}
\usepackage{graphicx}
\usepackage{gensymb}
\title{chapter4} 
\begin{document}
Problem 1. \\ \\ 
Prove that I, $R_{+}$, $R_{-}$, $R_{a}$, $R_{b}$, and $R_{c}$ are all the symmetries of the equilateral triangle.
[Hint: One way to do this is to label the three corners, as in Figure 4.2 A given symmetry operation carries A into the position formerly occupied by A, B, or C. If A $\rightarrow$ A, then either B $\rightarrow$ B and C $\rightarrow$ C, or else B $\rightarrow$ C and C $\rightarrow$ B. Take it from there.] \\ \\

Solution\\
\begin{figure}[h!]
  \includegraphics[width=\linewidth]{equilateraltriangle.png}
  \caption{equilateral triangle.}
  \label{fig:triangle1}
\end{figure}
Figure \ref{fig:triangle1} equilateral triangle.

Where,\\
$R_{+}$ : A clockwise rotation through 120$^{\circ}$\\
$R_{-}$ : A counterclockwise rotation through 120$^{\circ}$\\
$R_{a}$ : by flipping it about the vertical axis a.\\
$R_{b}$ : by flipping it about the vertical axis b.\\
I : Doing nothing.\\

Q. How to prove the symmetries? \\
A. By symmetry properties. \\ \\

Symmetry properties\\\\ 
1. Closure : $R_{i}$$R_{j}$ $\rightarrow$ first perform $R_{j}$ and then $R_{i}$ $\rightarrow$ and there exists $R_{k}$ = $R_{i}$$R_{j}$.\\
2. Identity : I$R_{i}$ = $R_{i}$I = $R_{i}$ .\\
3. Inverse : $R_{i}$$R^{-1}_{i}$ = $R^{-1}_{i}$$R_{i}$ = I .\\
4. Associativity : $R_{i}$($R_{j}$$R_{k}$) = ($R_{i}$$R_{j}$)$R_{k}$ .\\\\
1. $R_{-}$$R_{+}$ = I \\
$R_{a}$$R_{b}$$R_{c}$ = I \\ 
$R_{a}$$R_{b}$ = $R_{c}$ \\ 
2. Yes\\
3. Yes \\
4. Yes \\ 
(Not sure this is proper ?) \\ \\ \\ \\







\end{document}