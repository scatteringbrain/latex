\documentclass{report}
\title{chapter1} 
\begin{document}
Problem 1. \\ \\ If a charged particle is undeflected in passing through uniform crossed electric and magnetic fields E and B(mutually perpendicular and both perpendicular to the direction of motion), what is its velocity? if we now turn off the electric field, and the particle moves in an arc or radius R, what is its charge-to-mass ratio?


solution \\ 
1. lorentz force \begin{equation} F = q(E + v \times B) \end{equation} 
2. poynting vector (not used) \begin{equation} S = E \times H \end{equation} 

from the lorentz force, the velocity is derived by \begin{equation} qE = - v \times B \end{equation}   because the force is zero. (The E and B are perpendicular and the particle is undeflected.) 

Turn off E, and then moving in radius R, what is its charge-to-mass ratio?

motion of circle \begin{equation} F = mv^2/R = qvB 
q/m = (v(vel))^2/(v(vol)BR) \end{equation} \\ \\ \\ \\ \\

 
Problem 2. \\ \\
The mass of Yukawa's meson can be estimated as follows. When two protons in a nucleus exchange a meson (mass m), they must temporarily violate the conservation of energy by an amount \(mc^2\) (the rest energy of the meson). The Heisenberg uncertainty principle says that you may 'borrow' an energy \(\triangle E\), provide you 'pay it back' in a time \(\triangle t\) given by  \(\triangle E \triangle t = \hbar\) (where \(\hbar   \equiv \frac{h}{\pi} \)). In this case, we need to borrow \(\triangle E \) = \(mc^2\) long enough for the meson to make it from one proton to the other. It has to cross the nucleus (size \(r_0\)), and it travels, presumably, at some substantial fraction of the speed of light, so, roughly speaking, \(\triangle\)t = \(r_0\)/c. Putting all this together, we have
\[ m = \frac{\hbar}{2 r_0 c} \]

\end{document}
