\documentclass{report}
\usepackage{graphicx}
\usepackage{tikz}
\usepackage{tikz-feynman}

\title{chapter1} 
\begin{document}

Chapter 1 \\

Quantum Electrodynamics (QED) : Maxwell's equation , Dirac equation . \\ 

Feynman diagrams, Quantum mechanics, Relativity \\ 

Physical intuition $\rightarrow$ bottom-up approach $\rightarrow$  many gaps\\ 

Goal is the top-down approach \\ 

Cross section calculation \\ 

\[\frac{d\sigma}{d\Omega} = \frac{1}{64\pi^{2}E^{2}_{cm}\cdot |M|^2}\]
{\center( $\textnormal{CM scattering}$ ) \\}

For the QED, The `M' is not known.\\ 

The best we can do : Set M as a perturbation series of QED, and evaluate the first term. \\

The Feynman diagram $\rightarrow$ visualize the perturbation.\\

In QM perturbation theory, to first order, the amplitude is, \\
\begin{center}
$\langle$ $\textnormal{final state}$ \(|\ H_1\ |\) $\textnormal{initial state}$ $\rangle$ 

\feynmandiagram [vertical'=a to b] {
i1 [particle=\(\mu^{+}\)]
-- [fermion] a
-- [fermion] f1 [particle=\(\mu^{-}\)],
a -- [photon, edge label=\(\gamma\)] b,
i2 [particle=\(e^{-}\)]
-- [fermion] b
-- [fermion] f2 [particle=\(e^{+}\)],
}; \\
\end{center} 
This is the first order, but the hamiltonian can not mediate the two state, but gamma does it.\\
So, expand this equation to the next order with $\gamma$.\\ \\

For ($e^-$ + $e^+$ $\rightarrow$ $\mu^-$ + $\mu^+$), \\
\begin{center}
M $\sim$ $\langle$$\mu^+$$\mu^-$ $| H_1 |$ $\gamma$ $\rangle$$^\mu$ $\langle$$\gamma$$| H_1 |$$e^+$$e^-$$\rangle$$_\mu$ \\ 
\end{center}

1. External electron lines : $|$$e^{+}e^{-}$$\rangle$\\
2. External muon lines : $\langle$$\mu^{+}\mu^{-}$$|$\\
3. The vertices : $H_1$ \\
4. Internal photon line : $|$$\gamma$$\rangle$$\langle$$\gamma$$|$\\
5. The amplitude $M$ will be a Lorentz-invariant scalar as long as each states are 4-vectors.\\ \\
$\langle$$\gamma$$| H_1 |$$e^+$$e^-$$ \rangle$$_\mu$ $\rightarrow$ $H_1$ is related to $\gamma$ and electrons $\rightarrow$ the elements of the matrix is proportional to $e$.\\ 
Spin Orientations : \\
1. The electron and Muon $\rightarrow$ parallel spins. \\ 
2. The electron and Muon : "right handed", The positron and positive Muon : "left handed". \\ 
3. The electron and positron spins add up to one angular momentum in the +z direction. \\
The $H_1$ should conserve the angular momentum.\\
$\rightarrow$ photon's polarization vector : $\epsilon^\mu$ = (0,1,i,0)\\
\begin{figure}[!ht]
  \includegraphics[scale=0.25]{grifitth7_94.png}
\includegraphics[scale=0.25]{grifitth_7_159.png}  \caption{Grifitth chapter7}
  \label{fig:myfig1}
\end{figure}
%\includegraphics[scale=0.25]{grifitth7_94.png}
%\includegraphics[scale=0.25]{grifitth_7_159.png}
\end{document}
